\chapter{Fehlerdiskussion}
\section{dopplerfreie S�ttigungsspektroskopie}

Um die Leistung des Pumpstrahls �ber deren \(\frac{1}{e^2}\)-Breite in Intensit�t umzurechnen haben wir die Formel 
in \eqref{eq:intensitaetleistung} verwendet . Wie schon im dem theoretischen Teil beschrieben ist diese Formel jedoch eine grobe N�herung, da man zuerst von einer Gau�verteilten Intensit�t
ausgeht, aber dann annimmt die Intensit�t w�re konstant gleich der Amplitude \(I_0 \) der Gau�funktion.

Um sich dem theoretischen Model besser anzun�hern k�nnte man mit Blenden arbeiten, mit denen man die R�nde der Gau�funktion abschneiden kann und somit
den Unterschied zwischen gau�verteilter und konstanter Funktion zu senken.

Wir mussten bei der Aufnahme der 87F2-Linien mehrere Male die Messung wiederholen, da wir mit einem sehr starken Gau�verteilten Untergrund zu k�mpfen hatten, 
den wir uns ersteinmal nicht erkl�ren konnten. Er kam dem Effekt nahe, den wir im ersten Versuchsteil gemessen haben
und den wir mit dem Referenzstrahl eigentlich kompensieren wollten. Mysteri�se wurde es als sich der Effekt verst�rkte je h�her die Leistung
am Pumpstrahl eingestellt wurde. 


Es stellte sich heraus, dass Reflexionen des Pumpstrahls an den R�ndern der Rubidiumzelle daf�r verantwortlich waren. 

Wir haben das Ger�st in einem gr��eren Winkel schr�g zum Pumpstrahl, wie in \ref{fig:dopplerfrei} illustriert, gestellt
 damit die Reflexionen nicht mehr in die Photodiode trafen.

Dies hat die Messung erheblich vereinfacht und verbessert. Nat�rlich k�nnen wir empfehlen ein weniger spiegelndes Material zu verwenden,
nur ist das in der Arbeit mit optischen Gegenst�nden selbstverst�ndlich. Alternativ k�nnte man einen dritten Referenzstrahl bzw. 
die Reflexionen des Pumpstrahles in Abh�ngigkeit von dessen Leistung seperat aufnehmen und das als Hintergrundrauschen mitzubeachten.


